%  Copyright 2005 University of Karlsruhe
%  Authors:
%    Joshua LeVasseur <jtl@ira.uka.de>
%
%  vim:textwidth=70

%\documentclass[twoside,a4paper]{book}
\documentclass[twoside,a4paper]{report}
\usepackage{times}
\usepackage{color}
\usepackage{ifthen}
\usepackage{url}


\newcommand{\code}[1]{{\tt #1}}
\newcommand{\revision}{$ $Revision: 1.10 $ $}
%\newcommand{\cmd}[1]{\framebox{{\small \texttt{#1}}}}
%\newcommand{\cmd}[1]{\framebox{\parbox[t]{\textwidth}{\code{#1}}}}
\newcommand{\cmd}[1]{\code{#1}}

\newcommand{\Pistachio}{L4Ka::Pistachio}
\newcommand{\IDL}{IDL4}
\newcommand{\Marzipan}{Marzipan}

% abstract-data-type
\newcommand{\adt}[1]{\emph{#1}}

% ======================================================================
%
%    Title and copyright stuff
%
% ======================================================================

\title{\huge The Afterburner User Manual}

\author{Joshua LeVasseur\\
  System Architecture Group,\\
  Department of Computer Science,\\
  University of Karlsruhe, Germany
}

\date{Document \revision
  \\\today
}

\usepackage[dvips]{hyperref}
\hypersetup{
  pdfpagemode=none,
  bookmarksnumbered=true,
  pdfborder={0 0 0},
  pdftitle={The Afterburner User Manual},
  pdfauthor={Joshua LeVasseur, System Architecture Group, University of Karlsruhe},
  pdfkeywords={Pre-virtualization with Compiler Afterburning, User Manual, L4Ka},
}


\begin{document}
\pagestyle{empty}
\setcounter{page}{1}

%\frontmatter{}
\maketitle{}

\clearpage{}
\vspace*{\fill}
{\small

\noindent
Copyright \copyright\ 2005 by the System Architecture Group,
Department of Computer Science, University of Karlsruhe.

\vspace{1em}\noindent THIS MANUAL IS PROVIDED ``AS IS'' WITHOUT
ANY WARRANTIES, INCLUDING ANY WARRANTY OF MERCHANTABILITY,
NON-INFRINGEMENT, FITNESS FOR ANY PARTICULAR PURPOSE, OR ANY WARRANTY
OTHERWISE ARISING OF ANY PROPOSAL, SPECIFICATION OR SAMPLE.

\vspace{1em}\noindent Permission to copy and distribute verbatim
copies of this user manual in any medium for any purpose without fee
or royalty is hereby granted.  No right to create modifications or
derivates is granted by this license.  The L4Ka Team may make changes
to this user manual at any time, without notice.  The latest
revision of this document is available at \url{http://l4ka.org/}.

}


% ======================================================================
%
%    Table of Contents
%
% ======================================================================

\pagestyle{headings}
\clearpage
\setcounter{page}{3}

\tableofcontents
\cleardoublepage

% ======================================================================
%
%    Chapters
%
% ======================================================================

\raggedright
\setlength{\parskip}{1ex}

%\mainmatter{}
\chapter{Introduction}

In \emph{pre-virtualization}, the task of virtualization is multi-staged.  It
starts when compiling the guest OS, where tools prepare the guest OS
to be easily virtualized.  It ends at run-time, where an in-place
emulation layer, called the \emph{wedge}, emulates the CPU and platform
devices.

This manual explains how to build the tools necessary for
pre-virtualizing the guest OS.  It explains how to build the Wedge for
several virtualization environments, including the L4Ka Marzipan
environment and the Xen hypervisor.  It explains how to build the
complementary runtime components of the virtualization environments.
And this manual explains how to use the virtualization environments at
runtime.

\chapter{Quick Start}

\section{Automated Build}

The Afterburner CVS module includes a script to automatically
retrieve, patch, build, and prepare for boot all the necessary
components to test-drive a pre-virtualization environment.  The
scripts automatically build the support utilities, the L4Ka Marzipan
and Xen virtualization environments, and Linux 2.6 as a guest OS.
The Linux 2.6 binary can execute on raw hardware, on L4, or on Xen.

To use the build automation, first retrieve the Afterburner project
from the CVS repository (see Appendix~\ref{a:cvs} for information
about the CVS repository).  It will be fetched into a directory called
\code{afterburner}.  Second, change into the directory
\code{afterburner/patches} and execute \cmd{make world}.  It requires
the \code{wget} or \code{curl} utilities to retrieve remote archives
from the Internet (along with an Internet connection).

The output is copied into a boot directory, by default
\code{afterburner/boot}, and prepared for booting via GRUB.  The GRUB
menu file is \code{afterburner/boot/afterburn.lst}.  From here, it is
your responsibility to generate a bootable disk image for GRUB.  Linux
kernel modules are the exception; they are copied to
\code{lib/modules} in the installation root (see \code{INSTALL\_ROOT}
in Table~\ref{t:world_options}).

The build automation is configurable.  You can configure the options on the
command line to \code{make world}, or you can write the options into the
file \code{afterburner/patches/Makeconf.local}.  See
Table~\ref{t:world_options} for the list of configuration options, or
consult the \emph{Command line parameters} section of
\code{afterburner/patches/Makefile}.  Note that most paths must be
absolute.  For example: \cmd{make world BOOT\_DIR=/boot/afterburner
GRUB\_BOOT=(nd)/boot/afterburner}.

\begin{table}[hbt]
  \centering
  \begin{tabular}{l|p{6cm}}
  Parameter & Description \\
  \hline
  \code{SRCDIR} & The directory where source archives are unpacked. \\
  \code{ARCHIVE\_DIR} & The directory where source archives are stored.  \\
  \code{BUILD\_DIR} & The build directory. \\
  \code{BOOT\_DIR} & The directory where boot elements are stored,
  including the \code{afterburn.lst} file. \\
  \code{INSTALL\_ROOT} & Some utilities, such as binutils, are
  installed into a rich directory tree.  This is the root of the
  installation tree. Choose a non-critical directory.  The default is
  to use \code{afterburner/usr}. \\
  \code{VGA} & Set VGA=1 to favor the VGA console.  By default, the
  first serial port is used as the console. \\
  \code{GRUB\_BOOT} & The path used within the \code{afterburn.lst}
  file to locate the boot elements. \\
  \code{LINUX\_BOOT\_PARAMS} & The Linux boot parameters, such as the
  root file system, specified in the \code{afterburn.lst} file.  \\
  \code{VMSIZE} & The size of the virtual machine, in kilobytes.  Used
  in the \code{afterburn.lst} file. \\
  \end{tabular}
  \caption{\label{t:world_options}The configuration options for
  \code{make world}.  Most paths must be absolute paths.}
\end{table}

\section{Pre-built Binaries}

You need GRUB to boot the pre-built binaries.  Copy the pre-built
binaries into a GRUB boot directory, along with the included
\code{afterburn.lst} GRUB configuration file.  Modify the
\code{afterburn.lst} file to suit your environment.  Choose
appropriate boot parameters for Linux 2.6.  Choose a valid size for
the virtual machine.  And set the boot paths to the location of the
boot elements.


% ======================================================================
%
%    Chapter: Building Support Tools
%
% ======================================================================
\chapter{Building the Support Tools}

The technique of pre-virtualization with compiler afterburning
performs automated instruction transformations on the assembler files
of the guest OS, whether the assembler files are hand-written or
are generated by the compiler from a higher-level language.  The
afterburning requires support tools to carry out the automated code
transformations.  For the GNU development environment, the supported
option is an enhanced assembler.

\section{The GNU Assembler (binutils)}
\label{sec:binutils}

Obtain the source code for binutils and apply the Afterburner binutils
patch (see Appendix~\ref{a:patch} for more information on applying the
patches).  The patch is in the Afterburner CVS repository (see
Appendix~\ref{a:cvs} for information about the CVS repository) under
\code{afterburner/patches/binutils}.  The patch modifies the
GNU assembler, to add several fixes to the assembler's macro parser, an
extra operator to disable macro expansion, and some enhancements to
the macro parser.  All modifications are limited to the assembler's
macro facility.

When executing the assembler in combination with gcc 2.95, it is
sufficient to give priority in the search path for the modified
assembler over the standard assembler.  In newer versions of gcc, the
compiler chooses the assembler based on the location of the compiler,
and thus it becomes necessary to replace the original assembler, or to
build your own binutils and compiler located at some place under your
control, or to use the \code{-B} command line option of gcc to
override the location of the binutils.

To build the binutils (or just the assembler), with source directory
\code{\$SRCDIR}:
\begin{enumerate}
\item Choose a build directory, \code{\$BUILDDIR}, and an installation
directory, \code{\$INSTALLDIR}.
\item Create the empty build directory.
\item Change into the build directory.
\item Execute the configure script: \cmd{\$SRCDIR/configure
--prefix=\$INSTALLDIR}.  If building for a cross-compilation
environment, additionally specify the target environment, such as
\code{--target=i386-elf}.
\item Either build the entire binutils project, \cmd{make all},
or build only the assembler, \cmd{make all-gas}.
\item When the build succeeds, it generates a new assembler named
\code{\$BUILDDIR/gas/new-as}.  To use this binary, create a symbolic
link to this file using the name \code{as} and add to your search path
so that it is chosen before the normal \code{as}.  Alternatively,
execute \cmd{make install-gas} to install the \code{as} binary in
your chosen installation location, and ensure that it is in your
search path.  But if using a newer gcc version, such as gcc 3.3.4, gcc
will not search your path for \code{as}, but will instead choose the
\code{as} binary that is installed in close proximity to gcc.  Thus
build and install the entire binutils (\cmd{make install}).
\end{enumerate}

\section{The GNU C Compiler (gcc)}

To use your newly built assembler, it may be sufficient to give
it priority in your shell's search path.  For newer versions of gcc,
one needs to explicitly tell gcc where to find the assembler, using
the gcc \code{-B} command line option.  Or alternatively, you can
build a gcc that is installed at the same location as the assembler.
Some rough directions for building a gcc, with source directory
\code{\$GCCSRC}:
\begin{enumerate}
\item Choose a build directory, \code{\$GCCBUILD}.
\item Use the same installation directory used for configuring the
binutils, \code{\$INSTALLDIR}.
\item Create the empty build directory.
\item Change into the build directory.
\item Execute the configure script: \cmd{\$GCCSRC/configure
--prefix=\$INSTALLDIR --with-gnu-as --with-gnu-ld}.  If building for a
cross-compilation environment, additionally specify the target
environment, such as \code{--target=i386-elf}.
\item Execute \cmd{make}.
\item Execute \cmd{make install}
\end{enumerate}


% ======================================================================
%
%    Chapter: Building the Virtualization Environment
%
% ======================================================================
\chapter{Building the Virtualization Environment}

The Wedge is the runtime virtualization component that complements
pre-virtualization.  The Wedge emulates the CPU and platform devices
used by the guest OS.  Typically, the Wedge executes within the
address space of the guest OS to maximize performance.  The Wedge is
responsible for loading the guest OS, rewriting the
virtualization-sensitive instructions of the guest OS, and for
starting the guest OS.

\section{The L4Ka Environment}

The L4Ka Wedge uses the \Pistachio{} microkernel, and the L4Ka
Marzipan virtualization environment.  The following sections describe
how to build each of the components.

\subsection{The \Pistachio{} Microkernel}

From the CVS repository, retrieve the \code{pistachio} module
(see Appendix~\ref{a:cvs} for information about the CVS repository).

To build \Pistachio{}:
\begin{enumerate}
\item Change into the \code{pistachio/kernel} source directory.
\item Choose a build directory, \code{\$BUILDDIR}, and ensure that it
doesn't yet exist.  It must be an absolute path.
\item Tell \Pistachio{} to create the build directory: 
\cmd{make BUILDDIR=\$BUILDDIR}.
\item Change into the build directory.
\item Configure the \Pistachio{} microkernel: \cmd{make menuconfig}
  \begin{enumerate}
  \item In the \emph{Hardware} submenu, choose the \emph{IA-32}
  architecture, disable \emph{Multiprocessor Support}, and in the 
  \emph{Hardware/Miscellaneous} submenu, disable the
  \emph{APIC+IOAPIC} option.
  \item In the \emph{Kernel} submenu, enable the fast IPC path.
  \end{enumerate}
\item Build the microkernel: \cmd{make -s}
\item Copy the microkernel binary, \code{ia32-kernel}, to your boot
directory.
\end{enumerate}

\subsection{The \Pistachio{} Utilities}

Of the utilities included with the \Pistachio{} microkernel, the
Marzipan environment requires sigma0 and kickstart.  Sigma0 is the
root memory manager.  And kickstart is the \Pistachio{} pre-boot
loader.

To build the \Pistachio{} utilities, where the \code{pistachio} CVS
module is \code{\$PISTACHIO\_SRC}:
\begin{enumerate}
\item Create a build directory, \code{\$BUILDDIR}.
\item Change into the build directory.
\item Configure the utilities: \cmd{\$PISTACHIO\_SRC/user/configure}.
      The default parameters will use the serial port as the debug and
      output console.  Consider specifying
      \code{--without-comport} to use the VGA console instead.  Ensure
      that you used the same output console for the \Pistachio{}
      microkernel to avoid confusion.
\item Build the utilities: \cmd{make -s}
\item Copy \code{serv/sigma0/sigma0} and
      \code{util/kickstart/kickstart} to your boot directory.
\end{enumerate}


\subsection{The Marzipan Resource Monitor}

From the CVS repository, retrieve the \code{marzipan} module
(see Appendix~\ref{a:cvs} for information about the CVS repository).

Marzipan uses a set of interface definitons.  To compile the interface
definitions, \IDL{} is required.

To build Marzipan:
\begin{enumerate}
\item Change into the \code{marzipan} source directory.
\item Choose a build directory, \code{\$BUILDDIR}, and ensure that it
doesn't yet exist.  It must be an absolute path.
\item Tell Marzipan to create the build directory: 
\cmd{make BUILDDIR=\$BUILDDIR}.
\item Change into the build directory.
\item Build Marzipan: \cmd{make -s}
\item The final binary is called \code{ia32-hypervisor}.  Copy it to
your boot directory.
\end{enumerate}


\subsection{The L4Ka Wedge}

Retrieve the Afterburner Wedge code from the CVS repository (see
Appendix~\ref{a:cvs} for information about the CVS repository).  The
CVS module is \code{afterburner/afterburn-wedge}.

The Wedge imports header files from the \Pistachio{} microkernel, and
imports interface definitions from the Marzipan project.  To compile
the interface definitions, \IDL{} is required.

To build the Wedge with pass-through device access:
\begin{enumerate}
\item If your gcc doesn't know the location of the enhanced binutils,
then set the environment variable \code{AFTERBURN\_BINUTILS} to the
location of the binutils's binaries created in~\ref{sec:binutils}.
\item Change into the \code{afterburner/afterburn-wedge} directory.
\item Choose a build directory, \code{\$BUILDDIR}, and ensure that it
doesn't yet exist.  It must be an absolute path.
\item Tell the Wedge to create the build directory: 
\cmd{make BUILDDIR=\$BUILDDIR}.
\item Change into the build directory.
\item Configure the Wedge: \cmd{make menuconfig}
  \begin{enumerate}
  \item Change the \emph{Wedge type} option to \emph{L4Ka}.
  \item Ensure that \emph{Architecture} is \emph{IA-32}.
  \item Ensure that \emph{SMP} is disabled.
  \item Avoid changing the options for the physical and virtual
  location of the Wedge; other values are untested.
  \item In the \emph{L4Ka wedge configuration} sub-menu, set the path
  to the Marzipan resource monitor IDL files, which are in the
  \code{interfaces} subdirectory of the \code{marzipan} CVS module.
  Also set the path to your build directory of the \Pistachio{}
  utilities.
  \item In the \emph{Device handling} sub-menu, ensure that
  \emph{Permit pass-through device access} is enabled.
  \end{enumerate}
\item Build the wedge: \cmd{make -s}
\item The final wedge binary is named \code{afterburn-wedge}.  Copy it
to your boot directory.
\end{enumerate}



\section{The Xen Environment}

The Xen Environment requires two components: the Xen hypervisor, and
the KaXen Afterburner Wedge.

\subsection{The Xen Hypervisor}

The Xen Wedge executes on a nearly-standard Xen hypervisor.  The Xen
Wedge doesn't yet support the IO-APIC interrupt controller, and so it
relies on a slightly patched Xen hypervisor.  The patch to the Xen
hypervisor disables the IO-APIC (and additionally enables
cross-compilation on Mac OS X).  Retrieve the official sources for
the Xen hypervisor, and apply the patch from
\code{afterburner/patches/xen} (see Appendix~\ref{a:patch} for more
information on applying the Afterburner patches).

To build the Xen hypervisor, change into the
\code{xen-2.0} directory, and execute \cmd{make xen}.
The final hypervisor binary is \code{xen-2.0/xen/xen.gz}; copy
the hypervisor binary to your boot directory.

\subsection{The KaXen Wedge}

Retrieve the Afterburner Wedge code from the CVS repository (see
Appendix~\ref{a:cvs} for information about the CVS repository).  The
CVS module is \code{afterburner/afterburn-wedge}.

To build the Wedge with pass-through device access:
\begin{enumerate}
\item If your gcc doesn't know the location of the enhanced binutils,
then set the environment variable \code{AFTERBURN\_BINUTILS} to the
location of the binutils's binaries created in~\ref{sec:binutils}.
\item Change into the \code{afterburner/afterburn-wedge} directory.
\item Choose a build directory, \code{\$BUILDDIR}, and ensure that it
doesn't yet exist.  It must be an absolute path.
\item Tell the Wedge to create the build directory: 
\cmd{make BUILDDIR=\$BUILDDIR}.
\item Change into the build directory.
\item Configure the Wedge: \cmd{make menuconfig}
  \begin{enumerate}
  \item Change the \emph{Wedge type} option to \emph{KaXen}.
  \item Ensure that \emph{Architecture} is \emph{IA-32}.
  \item Ensure that \emph{SMP} is disabled.
  \item In the \emph{KaXen wedge configuration} sub-menu, set the path
  to the Xen hypervisor public include directory.  By default, this
  option already points to the directory
  \code{afterburner/xen-2.0/xen/include/public}.
  \item In the \emph{Device handling} sub-menu, ensure that
  \emph{Permit pass-through device access} is enabled.
  \end{enumerate}
\item Build the wedge: \cmd{make -s}
\item The final wedge binary is named \code{afterburn-wedge}.  Copy it
to your boot directory.
\end{enumerate}

% ======================================================================
%
%    Chapter: Building the Guest OS
%
% ======================================================================
\chapter{Building the Guest OS}

It is necessary to use a modified tool chain while building the guest
OS.  The modified tool chain performs the pre-virtualization.
Additionally, to improve the performance of the pre-virtualization,
the guest OS must be configured to enable some pre-virtualization
hooks.

\section{Linux 2.6}

Retrieve the source code for Linux 2.6, and apply the Afterburner
Linux patch.  The patch is located in \code{afterburner/patches/linux}
(see Appendix~\ref{a:patch} for more information on applying
Afterburner patches).  The patch adds functionality to Linux,
including some pre-virtualization hooks, several manually added code
annotations, a modified build script, DMA address translation hooks,
and new configuration options.

You will likely build many different configurations of the Linux
kernel, and thus it is highly advisable to use Linux 2.6's new
capability to build in a directory that is not the source directory.
To do so, you still execute \code{make} from within the source
directory, but with a command line parameter \code{O=$<$output
directory$>$}.

The supported virtualization environments are incomplete in many ways,
and for example, do not support BIOS calls.  Thus disable BIOS
services.

To build a Linux with pass-through driver access, in the build
directory \code{\$BUILDDIR}:
\begin{enumerate}
\item If your gcc doesn't know the location of the enhanced binutils,
then set the environment variable \code{AFTERBURN\_BINUTILS} to the
location of the binutils's binaries created in~\ref{sec:binutils}.
\item Create the empty build directory.
\item Consider copying one of the Afterburner seed 
configuration files into the build directory, renaming to
\code{\$BUILDDIR/.config}.  The seed files are
located in \code{afterburner/afterburn-wedge/doc/linux-2.6}.
\item Configure Linux 2.6: \cmd{make O=\$BUILDDIR menuconfig}
  \begin{enumerate}
  \item \textbf{In the \emph{Bus options} menu, use the \emph{Direct}
  PCI access mode.  Using the BIOS for PCI access is not supported.}
  \item Traverse to the \emph{Afterburn} menu
  \item Enable the \emph{Afterburn pre-virtualization} option.
  \item Configure the path to the build directory of the Afterburn
  wedge.
  \item Enable support for L4Ka::Pistachio and Xen as desired.
  \item Enable support for pass-through device access.
  \end{enumerate}
\item Build Linux 2.6: \cmd{make O=\$BUILDDIR}
\item When running on a hypervisor, we don't rely on the standard,
final build output, such as \code{bzImage}.  We use the intermediate
\code{vmlinux} ELF binary, but after stripping it and compressing it.  
  \begin{enumerate}
  \item Change into the build directory.
  \item Strip the binary: \cmd{strip -o vmlinux.stripped vmlinux}.
  \item Compress the binary: \cmd{gzip -fc vmlinux.stripped $>$ vmlinuz}.
  \end{enumerate}
\item Copy the \code{\$BUILDDIR/vmlinuz} binary to the boot location.
\end{enumerate}

% ======================================================================
%
%    Chapter: Using the virtualization environments
%
% ======================================================================
\chapter{Using the Virtualization Environments}

\section{Thread Local Storage}

Note that neither virtualization environment supports Linux's thread
local storage (TLS).  Thus you must disable the TLS to successfully
boot Linux.  Your system has TLS if the directory \code{/usr/lib/tls}
exists.  To disable, rename the directory, such as to
\code{/usr/lib/tls.disabled}.

If you try to use TLS within the L4Ka
Marzipan environment, you might see the error message \emph{afterburn:
Unsupported exception from user-level ...} followed by
\emph{afterburn: VM panic.  Halting VM threads.}. 

\section{The L4Ka Marzipan Environment}

The Marzipan environment is booted and initialized using the GRUB boot
loader.  GRUB uses a configuration file, \code{menu.lst}, which
ties together all of the components for runtime.

An example GRUB configuration file:
\begin{verbatim}
title = L4Ka Marzipan
kernel = boot/l4ka/kickstart
module = boot/l4ka/pistachio
module = boot/l4ka/sigma0
module = boot/l4ka/marzipan
module = boot/l4ka/afterburn-wedge vmstart vmsize=256M \
         wedgeinstall=16M wedgesize=16M
module = boot/linux-2.6 root=/dev/hda1 console=tty console=ttyS0
\end{verbatim}

\section{The Xen Environment}

The Xen environment is booted and initialized using the GRUB boot
loader.  GRUB uses a configuration file, \code{menu.lst}, which
ties together all of the components for runtime.

Currently, the command line arguments for the Linux kernel must be
passed via the command line of the Afterburn Wedge.

An example GRUB configuration file:
\begin{verbatim}
title = Afterburnt Xen
kernel = boot/xen/xen.gz dom0\_mem=262144 com1=115200,8n1
module = boot/xen/afterburn-wedge root=/dev/hda1 console=tty \
         console=ttyS0
module = boot/linux-2.6
\end{verbatim}

The KaXen wedge includes an interactive debugger.  If you enabled the
debugger while building the KaXen wedge, then you can activate the
debugger by pressing \cmd{Ctrl-x}.  Press \cmd{h} or \cmd{?} to see
the list of commands.  Press \cmd{g} to resume execution of the guest
OS.  You can reboot the system by pressing \cmd{6}.


\section{The QEMU Emulator}

The QEMU emulator is useful for development and testing, especially
when cross-compiling on Mac OS X.  The automated build script will retrieve,
patch, and build QEMU.  The patch improves support for Mac OS X
(although it doesn't fix key-binding problems when using the graphics
console device, nor does it enable support for Mac OS X
10.4\footnote{Mac OS X 10.4 broke the \code{poll()} system call, which QEMU
uses for its event loop, thus breaking QEMU's serial console.}).  The
patch additionally makes the built-in tftp server more portable
between machines, by removing support for paths in tftp requests.
Instead, the tftp server converts all file requests, no matter the
path, into requests for files from the current working directory of
QEMU.  Doing so permits a single GRUB configuration to simultaneously
support real hardware (and a real tftp server), and QEMU.  But more
importantly, it avoids the need to generate a custom boot floppy image
containing the proper tftp path (generating disk images is very
unportable).  Instead, the automated build script will retrieve a boot
floppy from the l4ka.org web site, which already contains a GRUB boot
sector configured to contact QEMU's tftp server to retrieve a
\code{menu.lst} file.

To execute QEMU, change into the
the \code{afterburn/patches} directory and execute 
\cmd{make run-qemu}.  If your boot directory (as recognized by the
automated build script) doesn't contain a GRUB \code{menu.lst} file,
the automated build script will generate an example file.  And
additionally, the automated build script will fetch a GRUB boot
floppy, called \code{qemu-afterburner-floppy.img}, from the l4ka.org
web site.  This file enables you to immediately use QEMU, without
having to create a real hard disk image.  Of course, without a hard
disk image, and without a RAM disk image, Linux will only partially
boot.

Invoking QEMU via the \cmd{make run-qemu} command provides a fairly portable
configuration to QEMU.  It disables graphics.  It uses user-level
networking (and attempts to tie the host port 3022 to the
guest OS at port 22 for ssh, and assumes that you configure your guest OS to
use IP address 10.0.2.10).  It provides a small amount of RAM, which
is probably insufficient for the default GRUB configuration.  Either
you will have to change the QEMU configuration, or decrease the amount
of RAM allocated to a VM in the GRUB configuration file (via the
\code{VMSIZE} parameter of the automated build script).
Since the graphics supports is disabled, QEMU will instead 
activate the current terminal for serial console
I/O.  Thus if you configure your guest OS to use the serial console,
then the booting kernel will send its boot messages to the current
terminal, and will open a login prompt at the current terminal.  And
if any of the virtualization environments support an interactive
debugger, then it can be activated from the current console.  The
L4Ka::Pistachio kernel often uses the \cmd{Escape} key to activate the
kernel debugger.  The KaXen wedge uses \cmd{Ctrl-x} to activate the
debugger.  QEMU has a command console, activated by pressing
\cmd{Ctrl-a}, followed by \cmd{c}. The Xen hypervisor uses three successive 
presses of Ctrl-a to activate its debug console, which interferes with
QEMU, and so is actually enabled by pressing \cmd{Ctrl-a} followed by
\cmd{a}, the whole pattern repeated three times.

If everything works as planned, you should be able to retrieve the
Afterburn project from CVS, execute \cmd{make world VMSIZE=32768}, and 
then execute \cmd{make run-qemu}, and boot Linux to the point where it tries to
mount a root disk.



% ======================================================================
%
%    Appendixes
%
% ======================================================================

\appendix

\chapter{CVS Repository}
\label{a:cvs}

The L4Ka public read-only CVS repository is available via CVS's
pserver protocol at \code{:pserver:guest@cvs.l4ka.org:/public-cvs}.
The password is \code{guest}.  It is also browseable online via a
CVSweb interface.  Several top-level modules are available, see
Table~\ref{t:modules}.

\begin{table}[hbt]
  \centering
  \begin{tabular}{l|p{6cm}}
  Module & Description \\
  \hline
  \code{afterburner} & The Afterburner project. \\
  \code{marzipan} & The \Marzipan{} project provides the L4Ka 
  runtime virtualization environment. \\
  \code{pistachio} & The \Pistachio{} microkernel, the basis of the
  L4Ka virtualization environment. \\
  \end{tabular}
  \caption{\label{t:modules}The modules available from the L4Ka public CVS.}
\end{table}

To login to the CVS server, execute:
\cmd{cvs -d:pserver:guest@cvs.l4ka.org:/public-cvs login}

To retrieve a module, such as the \code{afterburner} module, execute:
\cmd{cvs -d:pserver:guest@cvs.l4ka.org:/public-cvs co afterburner}


\chapter{Afterburner Support Patches}
\label{a:patch}

The Afterburner interacts with externally provided projects, including
build tools, hypervisors, and guest OSs.  Currently, several of the
external projects require modifications for compatibility with the
Afterburner.  The modifications are distributed as patches, and are
available in the \code{patches} sub-directory of the
\code{afterburner} CVS module.

The patches apply to specific versions of their respective projects,
but may also successfully apply to newer versions.  The external
projects, their versions, and remote locations of the source archives
are listed in the \code{afterburner/patches/SOURCES} file.

The \code{Makefile} in the \code{afterburner/patches} directory can
automatically download the original source archives, and apply their
respective Afterburner patches.  Executing \cmd{make} in the
\code{afterburner/patches} directory prints a help screen, and lists
the possible targets that it can retrieve and patch.

The \code{Makefile} stores downloaded source archives in the directory
\code{afterburner/patches/archives}.  If a source archive already
exists in the directory, then it won't be re-downloaded.  Also, if the
\code{Makefile} detects that the source archive has already been
unarchived, then it won't repeat the operation.  By default, the
source packages are unarchived in \code{afterburner/extern}; you can
choose a different directory via the \code{SRCDIR}
command line parameter.

For example, to automatically retrieve and patch Linux 2.6, execute
\cmd{make patch-linux-2.6}.  If the specific version is Linux 2.6.9, then
the final directory will be \code{afterburner/extern/linux-2.6.9}.  To use a
different container directory, for example \code{/tmp}, then execute
\cmd{make patch-linux-2.6 SRCDIR=/tmp}, and then the final directory will be
\code{/tmp/linux-2.6.9}.

To manually apply a patch, change into the target source archive, and
execute the patch command with the parameters \code{-Np1}.

\end{document}


